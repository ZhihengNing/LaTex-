\def\allfiles{}
\documentclass[12pt, a4paper, oneside]{ctexbook}
\def\configPath{./config}
\def\basicPath{\configPath/basic}


% 在这里定义需要的包
\usepackage{amsmath}
\usepackage{amsthm}
\usepackage{amssymb}
\usepackage{graphicx}
\usepackage{mathrsfs}
\usepackage{enumitem}
\usepackage{geometry}
\usepackage[colorlinks, linkcolor=black]{hyperref}
\usepackage{stackengine}
\usepackage{yhmath}
\usepackage{extarrows}

\usepackage{fancyhdr}
\usepackage[dvipsnames, svgnames]{xcolor}
\usepackage{listings}
\usepackage{titlesec}

\usepackage[strict]{changepage} 
\usepackage{framed}
\usepackage{color}
% 在这里定义自己顺手的环境
\def\d{\mathrm{d}}
\def\R{\mathbb{R}}
\newcommand{\bs}[1]{\boldsymbol{#1}}
\newcommand{\ora}[1]{\overrightarrow{#1}}
% 几倍的垂直空白
\newcommand{\myspace}[1]{\par\vspace{#1\baselineskip}}
% 代表两个参数,其中第一个参数的默认值为0,增加高度用的
\newcommand{\xrowht}[2][0]{\addstackgap[.5\dimexpr#2\relax]{\vphantom{#1}}}
% 条件选择
\newenvironment{ca}[1][1]{\linespread{#1} \selectfont \begin{cases}}{\end{cases}}
% 行列式
\newenvironment{vx}[1][1]{\linespread{#1} \selectfont \begin{vmatrix}}{\end{vmatrix}}

\newcommand{\tabincell}[2]{\begin{tabular}{@{}#1@{}}#2\end{tabular}}
% 双斜杠 //
\newcommand{\pll}{\kern 0.56em/\kern -0.8em /\kern 0.56em}
% divF
\newcommand{\dive}[1][F]{\mathrm{div}\;\bs{#1}}
% rotA
\newcommand{\rotn}[1][A]{\mathrm{rot}\;\bs{#1}} 
\usepackage[strict]{changepage} 
\usepackage{framed}

\definecolor{greenshade}{rgb}{0.90,1,0.92}
\definecolor{redshade}{rgb}{1.00,0.88,0.88}
\definecolor{brownshade}{rgb}{0.99,0.95,0.9}
\definecolor{lilacshade}{rgb}{0.95,0.93,0.98}
\definecolor{orangeshade}{rgb}{1.00,0.88,0.82}
\definecolor{lightblueshade}{rgb}{0.8,0.92,1}
\definecolor{purple}{rgb}{0.81,0.85,1}

\theoremstyle{definition}
\newtheorem{myDefn}{\indent Definition}[section]
\newtheorem{myLemma}{\indent Lemma}[section]
\newtheorem{myThm}[myLemma]{\indent Theorem}
\newtheorem{myCorollary}[myLemma]{\indent Corollary}
\newtheorem{myCriterion}[myLemma]{\indent Criterion}
\newtheorem*{myRemark}{\indent Remark}
\newtheorem{myProposition}{\indent Proposition}[section]

\newenvironment{formal}[2][]{%
	\def\FrameCommand{%
		\hspace{1pt}%
		{\color{#1}\vrule width 2pt}%
		{\color{#2}\vrule width 4pt}%
		\colorbox{#2}%
	}%
	\MakeFramed{\advance\hsize-\width\FrameRestore}%
	\noindent\hspace{-4.55pt}%
	\begin{adjustwidth}{}{7pt}\vspace{2pt}\vspace{2pt}}{%
		\vspace{2pt}\end{adjustwidth}\endMakeFramed%
}

\newenvironment{defn}{%
	\begin{formal}[Green]{greenshade}\vspace{-\baselineskip / 2}\begin{myDefn}}%
	{\end{myDefn}\end{formal}}

\newenvironment{thm}{%
	\begin{formal}[LightSkyBlue]{lightblueshade}\vspace{-\baselineskip / 2}\begin{myThm}}%
	{\end{myThm}\end{formal}}


\newenvironment{lemma}{%
	\begin{formal}[Plum]{lilacshade}\vspace{-\baselineskip / 2}\begin{myLemma}}%
	{\end{myLemma}\end{formal}}

\newenvironment{corollary}{%
	\begin{formal}[BurlyWood]{brownshade}\vspace{-\baselineskip / 2}\begin{myCorollary}}%
	{\end{myCorollary}\end{formal}}

\newenvironment{criterion}{%
	\begin{formal}[DarkOrange]{orangeshade}\vspace{-\baselineskip / 2}\begin{myCriterion}}%
	{\end{myCriterion}\end{formal}}

\newenvironment{rmk}{%
	\begin{formal}[LightCoral]{redshade}\vspace{-\baselineskip / 2}\begin{myRemark}}%
	{\end{myRemark}\end{formal}}

\newenvironment{proposition}{%
	\begin{formal}[RoyalPurple]{purple}\vspace{-\baselineskip / 2}\begin{myProposition}}%
	{\end{myProposition}\end{formal}}


\newtheorem{example}{\indent \color{SeaGreen}{Example}}[section]
\renewcommand{\proofname}{\indent\textbf{\textcolor{TealBlue}{Proof}}}
\newenvironment{solution}{\begin{proof}[\indent\textbf{\textcolor{TealBlue}{Solution}}]}{\end{proof}}

% \renewenvironment{proof}{\indent\textcolor{SkyBlue}{\textbf{proof.}}\;}{\qed\par}
% \newenvironment{solution}{\indent\textcolor{SkyBlue}{\textbf{Solution.}}\;}{\qed\par}

\definecolor{mygreen}{rgb}{0,0.6,0}
\definecolor{mygray}{rgb}{0.5,0.5,0.5}
\definecolor{mymauve}{rgb}{0.58,0,0.82}

% 图形文件搜索路径
\graphicspath{ {cover/figure/},{figure/},{chap1/figure/},{chap2/figure/} }

% 让\section标题靠左对齐
\titleformat{\section}[block]{\normalfont\Large\bfseries}{\chinese{section}、}{0pt}{}
% 设置行距
\linespread{1.6}

\geometry{
    top=25.4mm, 
    bottom=25.4mm, 
    left=20mm, 
    right=20mm, 
    headheight=2.17cm, 
    headsep=4mm, 
    footskip=12mm
}
% 设置列表环境
\setenumerate[1]{itemsep=5pt,partopsep=0pt,parsep=\parskip,topsep=5pt}
\setitemize[1]{itemsep=5pt,partopsep=0pt,parsep=\parskip,topsep=5pt}
\setdescription{itemsep=5pt,partopsep=0pt,parsep=\parskip,topsep=5pt}

\lstset{
    language=Mathematica,
    basicstyle=\tt,
    breaklines=true,
    keywordstyle=\bfseries\color{NavyBlue}, 
    emphstyle=\bfseries\color{Rhodamine},
    commentstyle=\itshape\color{black!50!white}, 
    stringstyle=\bfseries\color{PineGreen!90!black},
    columns=flexible,
    numbers=left,
    numberstyle=\footnotesize,
    frame=tb,
    breakatwhitespace=false,
} 
\title{\heiti 武汉理工大学研究生考试试卷(A卷)}
\renewcommand\thesection{\arabic {section}}
\begin{document}

\maketitle
\section{填空题}
\begin{enumerate}
\item $\bs{A}^T$,$n-r$
\item $\dim(\bs{V}_1+\bs{V}_2)=\dim(\bs{V}_1)+\dim(\bs{V}_2)-\dim(\bs{V}_1 \cap \bs{V}_2)=2+3-1=4$
\item 先对矩阵进行$\rm{LR}$分解,
$\bs{A}=\begin{pmatrix}
    -1 & 0&1\\
    1 & 2 & 0\\
    -1 &0 &1
\end{pmatrix}=\bs{L}\bs{R}=\begin{pmatrix}
    1 & 0&0\\
    -1 & 1 & 0\\
    1 &0 &1
\end{pmatrix}\begin{pmatrix}
    -1 & 0&1\\
    0 & 2 & 1\\
    0 &0 &0
\end{pmatrix}$

接着对矩阵$\bs{L}$进行列分块$\bs{L}=[\bs{\alpha}_1,\bs{\alpha}_2,\bs{\alpha_3}]$,
Schmidt正交化,得:

$\bs{\beta}_1=\bs{\alpha}_1=[1,-1,1]^T,
\bs{\beta}_2=\bs{\alpha}_2-\frac{(\bs{\alpha_2},\bs{\beta}_1)}{(\bs{\beta}_1,\bs{\beta}_1)} \bs{\beta}_1=\bs{\alpha}_2+\frac{1}{3}\bs{\beta}_1=[\frac{1}{3},\frac{2}{3},\frac{1}{3}]^T,
\bs{\beta}_3=\bs{\alpha}_3-\frac{(\bs{\alpha_3},\bs{\beta}_2)}{(\bs{\beta}_2,\bs{\beta}_2)} \bs{\beta}_2
-\frac{(\bs{\alpha_3},\bs{\beta}_1)}{(\bs{\beta}_1,\bs{\beta}_1)} \bs{\beta}_1
=\bs{\alpha}_3 -\frac{1}{2}\bs{\beta}_2-\frac{1}{3}\bs{\beta}_1=[-\frac{1}{2},0,\frac{1}{2}]^T$

对结果单位化,$[\bs{\gamma}_1,\bs{\gamma}_2,\bs{\gamma}_3]=
[\frac{\bs{\beta}_1}{||\bs{\beta}_1||},\frac{\bs{\beta}_2}{{||\bs{\beta}_2||}},\frac{\bs{\beta}_3}{{||\bs{\beta}_3||}}]
=[\frac{\bs{\beta}_1}{\sqrt{3}},\frac{\bs{\beta}_2}{{\sqrt{\frac{2}{3}}}},\frac{\bs{\beta}_3}{\sqrt{\frac{1}{2}}}]
=\begin{pmatrix}
   \frac{1}{\sqrt{3}}&\frac{1}{\sqrt{6}} & -\frac{1}{\sqrt{2}}\\
   \frac{-1}{\sqrt{3}} &\frac{2}{\sqrt{6}} &0\\
   \frac{1}{\sqrt{3}} &\frac{1}{\sqrt{6}} & -\frac{1}{\sqrt{2}}
\end{pmatrix}$


于是,$[\bs{\alpha}_1,\bs{\alpha}_2,\bs{\alpha}_3]=
[\bs{\gamma}_1,\bs{\gamma}_2,\bs{\gamma}_3]
\begin{pmatrix}
    \sqrt{3} & 0&0\\
    0 & \sqrt{\frac{2}{3}} & 0\\
    0 &0 &\sqrt{\frac{1}{2}}
\end{pmatrix}
\begin{pmatrix}
    1 & -\frac{1}{3}&\frac{1}{3}\\
    0 & 1 & \frac{1}{2} \\
    0 & 0&1
\end{pmatrix}$

\begin{align*}
\bs{L}\bs{R}&=
\begin{pmatrix}
    \frac{1}{\sqrt{3}}&\frac{1}{\sqrt{6}} & -\frac{1}{\sqrt{2}}\\
    \frac{-1}{\sqrt{3}} &\frac{2}{\sqrt{6}} &0\\
    \frac{1}{\sqrt{3}} &\frac{1}{\sqrt{6}} & -\frac{1}{\sqrt{2}}
 \end{pmatrix}
 \begin{pmatrix}
    \sqrt{3} & 0&0\\
    0 & \sqrt{\frac{2}{3}} & 0\\
    0 &0 &\sqrt{\frac{1}{2}}
\end{pmatrix}
\begin{pmatrix}
    1 & -\frac{1}{3}&\frac{1}{3}\\
    0 & 1 & \frac{1}{2} \\
    0 & 0&1
\end{pmatrix}
\begin{pmatrix}
    -1 & 0&1\\
    0 & 2 & 1\\
    0 &0 &0
\end{pmatrix}\\
&=
\begin{pmatrix}
    \frac{1}{\sqrt{3}}&\frac{1}{\sqrt{6}} & -\frac{1}{\sqrt{2}}\\
    \frac{-1}{\sqrt{3}} &\frac{2}{\sqrt{6}} &0\\
    \frac{1}{\sqrt{3}} &\frac{1}{\sqrt{6}} & -\frac{1}{\sqrt{2}}
 \end{pmatrix}
\begin{pmatrix}
    -\sqrt{3} & -\frac{2}{\sqrt{3}} &\frac{2}{\sqrt{3}}\\
    0 & \frac{2\sqrt{6}}{3} & \frac{\sqrt{6}}{3}\\
    0 &0 &0
\end{pmatrix}
\end{align*}

\item 3 , 3 ,3
\item $(\lambda-2)^3$
\end{enumerate}

\section{}
\begin{enumerate}
\item \begin{align*}
|\lambda \bs{E}-\bs{A}|=\begin{vmatrix}
    \lambda-3 &0 &-8 \\
    -3 & 1+\lambda & -6 \\
    2 &0 &\lambda +5
\end{vmatrix}=(\lambda+1)^3
\end{align*}
所以特征值$\lambda_1=\lambda_2=\lambda_3=-1$

行列式因子为$D_1=1,D_2=1+\lambda,D_3=(1+\lambda)^3$

不变因子为$d_1=1,d_2=1+\lambda,d_3=(1+\lambda)^2$

初等因子为$1+\lambda,(1+\lambda)^2$

\item Jordan标准形
$\bs{J}=\begin{pmatrix}
    -1 & 0 & 0\\
    0& -1 & 1 \\
    0& 0 & -1
\end{pmatrix}$,Smith标准形$\bs{S}=\begin{pmatrix}
    1 & 0 & 0\\
    0& 1+\lambda & 0 \\
    0& 0 & (1+\lambda)^2
\end{pmatrix}$

\item 最小多项式$m_{\bs{A}}(\lambda)=(\lambda+1)^2$
\end{enumerate}

\section{}
\begin{enumerate}
\item 只需验证其满足乘法和加法的八条公理,并且对加法和数乘法封闭
\item 由于存在$a_0,a_1,a_3$三个变量,自由度为3,则$\dim{(F[t]_3)}=3$,
容易看出$(e_1,e_2,e_3)=(1,t,t^2)$构成了$F[t]_3$中的一组基。于是:
\begin{align*}
\mc{T}(e_1)&=2=2e_1 \\
\mc{T}(e_2)&=1+2t=e_1+2e_2 \\
\mc{T}(e_2)&=2t+2t^2=2e_2+2e_3
\end{align*}
即$\mc{T}(e_1,e_2,e_3)=(e_1,e_2,e_3)
\begin{pmatrix}
2 & 1 & 0 \\
0 & 2 & 2 \\
0 & 0 & 2      
\end{pmatrix}$

\item $\dim(\rm{Ker}\mc{T})=0$,所以其基为$\bs{0}$

$\dim(\rm{Im}\mc{T})=3$,所以其基为$(2e_1,e_1+2e_2,2e_2+2e_3)$

\end{enumerate}
\section{}
$\bs{A}-\lambda\bs{E}=\begin{pmatrix}
3-\lambda& 0& 8 \\
3& -1-\lambda& 6 \\
-2& 0 &-5-\lambda
\end{pmatrix}$

特征值为$\lambda_1=\lambda_2=\lambda_3=-1$,
由于其最小多项式$m_{\bs{A}}(\lambda)=(\lambda+1)^2$,
不妨设$p(\lambda)=a_0+a_1\lambda$,可得下列方程组:
\begin{align*}
p(-1)&=a_0-a_1=e^{-t} \\
p'(-1)&=a_1=te^{-t}
\end{align*}
解得$a_0=(t+1)e^{-t},a_1=te^{-t}$。则:

$e^{\bs{A}t}=p(\bs{A})=
(t+1)e^{-t}\bs{E}+te^{-t}\bs{A}=\begin{pmatrix}
    (4t+1)e^{-t} & 0 & 8te^{-t} \\
    3te^{-t} & te^{-t} & 6te^{-t} \\
    -2te^{-t}& 0 &(1-4t)e^{-t}
\end{pmatrix}    $

微分方程的解为:

$\bs{x}=e^{\bs{A}t}\bs{x}(0)+e^{\bs{A}t} \int_{0}^t e^{-\bs{A}u}\bs{b}(t)du
=\begin{pmatrix}
    (4t+1)e^{-t} & 0 & 8te^{-t} \\
    3te^{-t} & te^{-t} & 6te^{-t} \\
    -2te^{-t}& 0 &(1-4t)e^{-t}
\end{pmatrix} \begin{pmatrix}
    1 \\
    0\\
    0
\end{pmatrix}=
\begin{pmatrix}
    (4t+1)e^{-t}  \\
    3te^{-t}  \\
    -2te^{-t}
\end{pmatrix}$

\section{}
\begin{enumerate}
\item 只需验证子空间$\bs{W}$对加法和数乘封闭。
\item 显然,$\dim(\bs{W})=2$,
则$\dim(\bs{W}^{\perp})=4-2=2$。

取$\bs{e}_1=\frac{1}{\sqrt{3}}\begin{pmatrix}
    -1 & 1\\
    0 & 0
\end{pmatrix},\bs{e}_2=\frac{1}{\sqrt{33}}\begin{pmatrix}
    2 & 1\\
    3 & 3
\end{pmatrix}$,显然这构成了在$\bs{W}$上的标准正交基。

取$\bs{e}_3=\frac{1}{\sqrt{10}}\begin{pmatrix}
    2 & 1\\
    -2 & 0
\end{pmatrix},\bs{e}_4=\frac{1}{\sqrt{110}}\begin{pmatrix}
    4 & 2\\
    6 & -5
\end{pmatrix}$,显然这构成了在$\bs{W}^{\perp}$上的标准正交基。
\end{enumerate}

\section{}
\begin{enumerate}
\item 
\begin{align*}
\bs{A}=\begin{pmatrix}
    1& 0 &1 \\
    0& 1 & 1\\
    -2 & 0& -2
\end{pmatrix}=\bs{B}\bs{C}=\begin{pmatrix}
    1& 0  \\
    0& 1 \\
    -2 & 0
\end{pmatrix}\begin{pmatrix}
    1& 0  &1\\
    0& 1 &1\\
\end{pmatrix}
\end{align*}

\item  
\begin{align*}
    \bs{A}^+&=\bs{C}^T(\bs{C}\bs{C}^T)^{-1}(\bs{B}^T\bs{B})^{-1}\bs{B}^T \\
    &=\begin{pmatrix}
        1& 0  \\
        0& 1 \\
        1 & 1
    \end{pmatrix}
    \frac{1}{3}
    \begin{pmatrix}
        2& -1  \\
        -1& 2 \\
    \end{pmatrix}
    \begin{pmatrix}
        \frac{1}{5}& 0  \\
        0& 1 \\
    \end{pmatrix}
    \begin{pmatrix}
        1& 0 &-2 \\
        0& 1 & 0\\
    \end{pmatrix}\\
    &=\begin{pmatrix}
        \frac{2}{15}& -\frac{1}{3} &-\frac{4}{15} \\
        -\frac{1}{15}& \frac{2}{3} & \frac{2}{15}\\
        \frac{1}{15} & \frac{1}{3}& -\frac{2}{15}
    \end{pmatrix}
\end{align*}
\item  若有解,则$\bs{x}_0=\bs{A}^+\bs{b}$显然是方程$\bs{A}\bs{x}=\bs{b}$的一个解,
且必满足$\bs{A}\bs{A}^+\bs{b}=\bs{b}$,经验证,
$\bs{A}\bs{A}^+\bs{b}=\bs{b}$这样的等式不成立,则$\bs{A}\bs{x}=\bs{b}$不相容。

\item \begin{align*}
\bs{x}&=\bs{A}^{+}\bs{b}+(\bs{E}-\bs{A}^+\bs{A})\bs{y}\\
    &=\begin{pmatrix}
        \frac{2}{15}& -\frac{1}{3} &-\frac{4}{15} \\
        -\frac{1}{15}& \frac{2}{3} & \frac{2}{15}\\
        \frac{1}{15} & \frac{1}{3}& -\frac{2}{15}
    \end{pmatrix} \begin{pmatrix}
        1 \\
        2 \\
        0
    \end{pmatrix}+\begin{pmatrix}
        \frac{1}{3}& \frac{1}{3} &-\frac{1}{3} \\
        \frac{1}{3}& \frac{1}{3} & -\frac{1}{3}\\
        -\frac{1}{3} & -\frac{1}{3}& \frac{1}{3}
    \end{pmatrix} \begin{pmatrix}
        y_1 \\
        y_2 \\
        y_3
    \end{pmatrix}\\
    &=\begin{pmatrix}
        -\frac{8}{15}+\frac{1}{3}y_1+\frac{1}{3}y_2-\frac{1}{3}y_3 \\
        \frac{19}{15}+\frac{1}{3}y_1 +\frac{1}{3}y_2-\frac{1}{3}y_3\\
        \frac{11}{15}-\frac{1}{3}y_1 -\frac{1}{3}y_2 +\frac{1}{3}y_3
    \end{pmatrix}
\end{align*}
其中$\bs{y}=[y_1,y_2,y_3]^T\in \R^3$。

\item 极小范数二乘解为:
\begin{align*}
\bs{x}_0&=\bs{A}^{+}\bs{b}\\
&=\begin{pmatrix}
    -\frac{8}{15} \\
    \frac{19}{15}\\
    \frac{11}{15}
\end{pmatrix}
\end{align*}
\end{enumerate}

\bibliographystyle{plain}
\bibliography{literature/literature}
\end{document}