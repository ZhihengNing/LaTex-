\usepackage[strict]{changepage} 
\usepackage{framed}

\definecolor{greenshade}{rgb}{0.90,1,0.92}
\definecolor{redshade}{rgb}{1.00,0.88,0.88}
\definecolor{brownshade}{rgb}{0.99,0.95,0.9}
\definecolor{lilacshade}{rgb}{0.95,0.93,0.98}
\definecolor{orangeshade}{rgb}{1.00,0.88,0.82}
\definecolor{lightblueshade}{rgb}{0.8,0.92,1}
\definecolor{purple}{rgb}{0.81,0.85,1}
\theoremstyle{definition}

% 这里的记忆方式为第一个大括号就是名称,第二个大括号是标识,即显示在pdf上的文字是什么,第三个就是与什么同步,
%同步的意思就是,假设我与section同步,比如我这里section是1.1,那么我的同步第一个就是1.1.1,第二个就是1.1.2
%假设我与chapter同步,比如我这里chapter是1,那么我的同步第一个就是1.1,第二个就是1.2
% 写section的话就是与section同步
\newtheorem{myDefn}{\indent 定义}[section]
% 可以设置与section同步,这里我认为引理最好和section同步,如果是多章节的话与chapter同步也是可以的
% \newtheorem{myLemma}{\indent 引理}[section]
\newtheorem{myLemma}{\indent 引理}[chapter]
\newtheorem{myThm}{\indent 定理}[myLemma]
\newtheorem{myCorollary}{\indent 推论}[myLemma]
\newtheorem{myCriterion}{\indent 准则}[myLemma]
\newtheorem*{myRemark}{\indent 注}
\newtheorem{myProposition}{\indent 命题}[section]
\newtheorem{example}{\indent \color{SeaGreen}{例}}[section]
\newtheorem{question}{\color{SeaGreen}{题}}[chapter]

% 第一个大括号是环境名称,第二个中括号是参数个数,
% 第三个中括号是首参数的默认值,第四个大括号是环境前定义,第五个大括号是环境后定义
\newenvironment{formal}[2][]
{%
    \def\FrameCommand{%
        \hspace{1pt}%
        {\color{#1}\vrule width 2pt}%
        {\color{#2}\vrule width 4pt}%
        \colorbox{#2}%
    }%
    \MakeFramed{\advance\hsize-\width\FrameRestore}%
    \noindent\hspace{-4.55pt}%
    \begin{adjustwidth}{}{7pt}\vspace{2pt}\vspace{2pt}
}
{
    \vspace{2pt}\end{adjustwidth}\endMakeFramed%
}

\newenvironment{defn}
{\begin{formal}[Green]{greenshade}\vspace{-\baselineskip / 2}\begin{myDefn}}
{\end{myDefn}\end{formal}}

\newenvironment{thm}
{\begin{formal}[LightSkyBlue]{lightblueshade}\vspace{-\baselineskip / 2}\begin{myThm}}
{\end{myThm}\end{formal}}

\newenvironment{lemma}
{\begin{formal}[Plum]{lilacshade}\vspace{-\baselineskip / 2}\begin{myLemma}}
{\end{myLemma}\end{formal}}

\newenvironment{corollary}
{\begin{formal}[BurlyWood]{brownshade}\vspace{-\baselineskip / 2}\begin{myCorollary}}
{\end{myCorollary}\end{formal}}

\newenvironment{criterion}
{\begin{formal}[DarkOrange]{orangeshade}\vspace{-\baselineskip / 2}\begin{myCriterion}}
{\end{myCriterion}\end{formal}}

\newenvironment{rmk}
{\begin{formal}[LightCoral]{redshade}\vspace{-\baselineskip / 2}\begin{myRemark}}
{\end{myRemark}\end{formal}}

\newenvironment{proposition}
{\begin{formal}[RoyalPurple]{purple}\vspace{-\baselineskip / 2}\begin{myProposition}}
{\end{myProposition}\end{formal}}


% \renewenvironment{proof}{\indent\textcolor{SkyBlue}{\textbf{证明.}}\;}{\qed\par}
% \newenvironment{solution}{\indent\textcolor{SkyBlue}{\textbf{解.}}\;}{\qed\par}

\renewcommand{\proofname}{\textbf{\textcolor{TealBlue}{证明}}}
\newenvironment{solution}{\begin{proof}[\textbf{\textcolor{TealBlue}{解}}]}{\end{proof}}
