% 在这里定义自己顺手的环境
\def\d{\mathrm{d}}
\def\R{\mathbb{R}}
\newcommand{\bs}[1]{\boldsymbol{#1}}
\newcommand{\mc}[1]{\mathcal{#1}}
\newcommand{\ora}[1]{\overrightarrow{#1}}
% 几倍的垂直空白
\newcommand{\myspace}[1]{\par\vspace{#1\baselineskip}}
% 代表两个参数,其中第一个参数的默认值为0,增加高度用的
\newcommand{\xrowht}[2][0]{\addstackgap[.5\dimexpr#2\relax]{\vphantom{#1}}}
% 条件选择
\newenvironment{ca}[1][1]{\linespread{#1} \selectfont \begin{cases}}{\end{cases}}
% 行列式
\newenvironment{vx}[1][1]{\linespread{#1} \selectfont \begin{vmatrix}}{\end{vmatrix}}

\newcommand{\tabincell}[2]{\begin{tabular}{@{}#1@{}}#2\end{tabular}}
% 双斜杠 //
\newcommand{\pll}{\kern 0.56em/\kern -0.8em /\kern 0.56em}
% divF
\newcommand{\dive}[1][F]{\mathrm{div}\;\bs{#1}}
% rotA
\newcommand{\rotn}[1][A]{\mathrm{rot}\;\bs{#1}} 